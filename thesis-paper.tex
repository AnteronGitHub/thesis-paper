\documentclass[officiallayout]{tktla}
%\documentclass[officiallayout,a4frame]{tktla}
\usepackage[latin1]{inputenc}
\usepackage{latexsym}
\usepackage{graphicx}

\usepackage{algpseudocode}
\usepackage{hyperref}
\usepackage{bytefield}
\usepackage{graphicx}
\usepackage{subcaption}

\newcommand{\pluseq}{\mathrel{+}=}

\title{Automated Software Configuration for Cloud Deployment}
\author{Antero Vainio}
\authorcontact{antero.vainio@helsinki.fi}
\pubtime{September}{2020}
\reportno{}
\isbnpaperback{}
\isbnpdf{}
\issn{}
\printhouse{}
\pubpages{7} % --- remember to update this!
% For monographs, the number of the last page of the list of references
% For article-based theses, the number of the last page of the list of
% references of the preamble part + the total number of the pages of
% the original articles and interleaf pages.
\supervisorlist{Lirim Osmani, Ashwin Rao, University of Helsinki, Finland}
\preexaminera{}
\preexaminerb{}
\custos{}
\generalterms{thesis, example, another example, still more examples,
  more and more examples}
\additionalkeywords{example, an example phrase with many words}
% Computing Reviews 1998 style
%\crcshort{A.0, C.0.0}
%\crclong{
%\item[A.0] Example Category
%\item[C.0.0] Another Example
%}
% Computing Reviews 2012 style
\crclong{
\item Example Category $\rightarrow$ And its subcategory $\rightarrow$ And sub-subcategory
\item Another Example  $\rightarrow$ And its subcategory
}
\permissionnotice{
    Thesis paper, to be published.
}

\newtheorem{theorem}{Theorem}[chapter]
\newenvironment{proof}{\noindent\textbf{Proof.} }{$\Box$}

\begin{document}

\frontmatter

\maketitle

\tableofcontents

\mainmatter

\chapter{Introduction}

As providing services over the Internet has become a common practice in the
modern society, maintenance of information systems has encountered many
challenges as well. While web-based applications have become more diverse,
resource-intensive, and complicated, they face higher expectations with respect
to usability, performance and security.

Often times industrial software development has strict deadlines to follow.
Many development practices are built around the idea of constant service
improvement. It means that a software product is rarely considered finished but
instead new features are being added to it and flaws being fixed in a priority
order.

As a result, programs and sometimes entire system architectures tend to require
frequent updates. Similarly to IT services automating repetitive tasks in hopes
of achieving reliability and cost-effectiveness, software development processes
aim to utilize automation whenever feasible.

Cloud enviroments offer on-demand computing resources for cloud consumers. When
deploying to cloud, a user can expect customized servers being provisioned
within seconds. It can be achieved with a combination of hypervisors and
programs preinstalled in virtual machines for instance. For further
configurations, additional software automation tools can be used.

Cloud environments are formed by physical host machines that lease virtualized
devices such as virtual CPUs (vCPU), logical volumes, or virtual network
devices. Users get typically charged for the time that they use some of these
resources. Servers can quickly be set up and down, preventing users from
having to pay for under-utilized servers.

For server provisioning, cloud providers do not typically have the same
luxuries as cloud users, since using virtualization is not optimal due to added
processing overhead. However when it comes to configuring hundreds or even
thousands of physical server computers, automation is once again seen as a
potential solution. Yet many traditional cloud administrators are still relying
on manual maintenance operations.

This thesis explores different popular methods for deploying and administering
cloud environments with help of software automation. Selected automation tools
and de facto methods for using them are compared. Questions \ref{fig:rqs} are
reflected during evaluation.

Deployed software is a combination OpenStack services. OpenStack is a
collective of open-source software projects for running cloud servers.
OpenStack is the most widely used plaftorm for creating private clouds. It
includes official repositories for various automated deployment methods. Most
of the methods provide a basis on which to build a solution customized for own
working environment.

\begin{figure}[t]
\centering
\begin{itemize}
  \item [RQ1] What are the key factors affecting the design of different
              deployment methods, and how do they differentiate from each
              other?
  \item [RQ2] What kind of features do different deployment methods offer?
              Which components are used in deployments?
  \item [RQ3] Who will benefit from the development of software automation
              tools, and why?
\end{itemize}
\caption{Research Questions}
\label{fig:rqs}
\end{figure}

\begin{thebibliography}{9}

\end{thebibliography}

\end{document}
